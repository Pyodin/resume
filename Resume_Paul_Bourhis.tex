\documentclass[10pt,a4paper]{moderncv}
\moderncvtheme[blue]{classic}
\usepackage[top=0.5cm, bottom=1cm, left=1.4cm, right=1.5cm]{geometry}

\setlength{\hintscolumnwidth}{2.8cm} % Largeur de la colonne pour les dates
\usepackage[utf8]{inputenc}% encodage, à modifier selon vos habitudes
\usepackage[french]{babel}% pour un document en français.
\usepackage{helvet}
\usepackage{graphicx}

%-----------------------------------------------------
% Une entête classique
\title{French DevOps - 3.5 years of experience \\  Looking for a new opportunity abroad}
\firstname{Paul}
\familyname{Bourhis}
\extrainfo{\href{https://www.linkedin.com/in/paul-bourhis-375557153/}{Linkedin link} 
	 \\ 27 years old }
\email{paul\_bourhis@yahoo.com}
\mobile{+33 6 99 29 45 22}
\address{49-51 Rue de Paris}{92110 Clichy - FRANCE}

\photo[54pt][0.8pt]{id1}
%-----------------------------------------------------
\begin{document}
\maketitle
%-----------------------------------------------------
\vspace{-3\baselineskip}
%-----------------------------------------------------
\section{Skills}

\subsection{DevOps}
\cvcomputer {Cloud:}{Azure (Certification Az-900) , AWS} { Monitoring: }
		{Logstash, Elasticsearch, Grafana, \\ Promotheus} 
\cvcomputer {Versioning:}{Git, Perforce}
		{ CI: }{Jenkins}
\cvcomputer { Administration: }{Ansible, Terraform}
		{ Container: }{Docker, Kubernetes} 

\subsection{Programming}
\cvcomputer { Language: }{Python, C/C++, Bash, Groovy, \LaTeX }
		{Data science: }{ Data analysis with pandas \newline Advanced digital signal processing }

\cvcomputer { Other: }{UI design with the Qt framework}
		{OS: }{Linux (Red Hat, Ubuntu)} 

\subsection{Languages}
\cvcomputer{English:}{Fluent and professional (TOEIC: 915)}{Spanish:}{Level C1 (several months spent in Spain)}


%-----------------------------------------------------

\section{Relevent experience}

\cventry{03/2021 - actual \\\includegraphics[width=0.7\hintscolumnwidth]{murex.png}}
{DevOps - Python Developer (1.5 years) }
{MUREX}
{\textsc{Paris}}
{}
{Developer and Devops in the MACS software team.
\begin{itemize}
\item CI Chanpion in a team of 10 developers
\item SRE on web services hosting the main C++ solution
\item Implementation of a stack of monitoring for services and a server cluster
\item Administration of Linux remote servers
\item Containerization of several application with Docker
\end{itemize}
}

\cventry{09/2019 - 02/2022 \includegraphics[trim=0 1 0 -6cm , width=0.7\hintscolumnwidth]{thaleslogo}}
{Embedded Software Engineer (2 years)}{THALES SIX GTS}
{\textsc{Paris}}
{}
{C++ Development of software components for SCORPIONS vehicles
\begin{itemize}
\item Implementation of data exchange between the vehicle and its equipment (C++)
\item Migration of the build chain to Cmake
\item Development of unit and functional tests (Jenkins and python)
\item Building of the test platform, and setting it up with cmake and our IDE
\end{itemize}
Development of a GUI intended to pilot CONTACT radio products
\begin{itemize}
\item Implementation of an SNMP client-server
\item Development of the GUI in Python with Qt tools
\end{itemize}
}

\cventry{04/2018  \\\includegraphics[width=0.5\hintscolumnwidth]{upv}}
{Internship Assistant Engineer (3 month)}{UNIVERSITAT POLITECNICA DE VALENCIA }
{}{}
{ C development of a robotic arm based on a STM32 board
\begin{itemize}
\item{Implementation of incremental encoders between the motors and the STM32 board}
\item{Management of sending and receiving data via UART link}
\item{Acquisition and processing of data by a python algorithm}
\end{itemize}
}


\cventry{05/2017}{Worker internship (2 months)}{EDISON ELECTRONICS}
{\textsc{Badalona (Spain)}}{}
{ As a worker in a PCB manufacturing company, I helped in the design and production of electronic cards.
}


\section{Education}

\cventry{2016 - 2019 \\\includegraphics[width=0.6 \hintscolumnwidth]{ensea2}}{Graduate Electrical Engineering, IT and Telecommications Shool}{at ENSEA}{\textsc{France}}{}{
Major:  Embedded Systems \newline Option: Computer, Automation and Internet of things }

\cventry{2013 - 2016}{Preparatory class PTSI-PT}{Lycée A.R-LESSAGE}{\textsc{France}}{}{Three-year undergraduate intensive course preparing for the competitive entrance examinations to French top-level "Grandes Écoles".
\newline Main subjects: Physics, Mathematics and Computer-science}

%-----------------------------------------------------

\section{Projects and activities}

\cventry{2022}{Project resume}{(Terraform / Azure)}{}{Provides a linux dev environment via terraform}{\url{https://github.com/Pyodin/azure-terraform-project-resume}}

\cventry{2015 - 2019}{Private teacher}{Mathematics and Physics}{}{}{}

\cventry{2018 - 2019}{University Project}{ENSEA}{Cross-development on a Linux i.MX6 card}{}{
\begin{itemize}
\item Configuration of a Linux kernel with Yocto
\item Acquisition and data transfer from an inertial unit by SPI, CAN and Wi-Fi link
\end{itemize}
}

\cventry{2017 - 2018}{Associations}{ENSEA}{}{Student board member, and organizer of the Integration Weekend}{}


%-----------------------------------------------------
\end{document}
